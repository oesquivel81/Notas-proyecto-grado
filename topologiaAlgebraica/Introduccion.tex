\documentclass[11pt]{article}

% ---------- Paquetes ----------
\usepackage[margin=1in]{geometry}
\usepackage{amsmath, amssymb, amsthm, mathtools}
\usepackage{enumitem}
\usepackage{hyperref}

% ---------- Macros útiles ----------
\newcommand{\R}{\mathbb{R}}
\newcommand{\id}{\mathrm{id}}
\newcommand{\norm}[1]{\left\lVert #1 \right\rVert}
\newcommand{\betti}{\beta}

\title{Nota de introducción: Caminos, Homotopía y su conexión con TDA}
\author{}
\date{}

\begin{document}
\maketitle

\tableofcontents

\section{Objetivo y notación}
El objetivo de esta nota es introducir:
\begin{itemize}[leftmargin=*]
  \item caminos en espacios topológicos,
  \item homotopía entre funciones (y entre caminos),
  \item relación de equivalencia y clases,
  \item una conexión básica con Topological Data Analysis (TDA) mediante complejos de Vietoris--Rips.
\end{itemize}

\medskip
\noindent Notación:
\begin{itemize}[leftmargin=*]
  \item $[0,1]$ es el intervalo unidad.
  \item $S^1=\{(x,y)\in\R^2: x^2+y^2=1\}$ es el círculo unitario.
  \item $\mathcal{C}(X,Y)$ denota el conjunto de funciones continuas $X\to Y$.
\end{itemize}

\section{Relación de equivalencia}
Sea $A$ un conjunto. Una relación $\sim$ sobre $A$ es una \textbf{relación de equivalencia} si para todo $a,b,c\in A$:
\begin{enumerate}[label=(\alph*), leftmargin=*]
  \item \textbf{Reflexiva:} $a\sim a$.
  \item \textbf{Simétrica:} si $a\sim b$, entonces $b\sim a$.
  \item \textbf{Transitiva:} si $a\sim b$ y $b\sim c$, entonces $a\sim c$.
\end{enumerate}

\subsection{Clases de equivalencia y cociente}
La \textbf{clase de equivalencia} de $a\in A$ es
\[
[a]=\{x\in A:\; x\sim a\}.
\]
El conjunto cociente es
\[
A/\!\sim \;=\;\{[a]: a\in A\}.
\]
Las clases de equivalencia forman una partición de $A$ (no se traslapan y cubren todo $A$).

\section{Caminos (paths)}
Sea $X$ un espacio topológico.

\subsection{Definición}
Un \textbf{camino} en $X$ es una función continua
\[
\gamma:[0,1]\to X.
\]
Decimos que $\gamma$ va de $a$ a $b$ si
\[
\gamma(0)=a,\qquad \gamma(1)=b.
\]

\subsection{Ejemplo en $\R^2$}
De $a=(0,0)$ a $b=(1,1)$:
\[
\gamma(t)=(t,t).
\]
Un \textbf{lazo} es un camino cerrado: $\gamma(0)=\gamma(1)$.

\section{Homotopía entre funciones}
Sean $X,Y$ espacios topológicos y $f,g:X\to Y$ funciones continuas.

\subsection{Definición}
Decimos que $f$ es \textbf{homotópica} a $g$ (se escribe $f\simeq g$) si existe una función continua
\[
H:X\times[0,1]\to Y
\]
tal que
\[
H(x,0)=f(x),\qquad H(x,1)=g(x)\quad \text{para todo }x\in X.
\]
Intuición: $H$ es una deformación continua de $f$ a $g$.

\section{La homotopía es una relación de equivalencia}
Consideremos la relación $\simeq$ sobre $\mathcal{C}(X,Y)$ dada por homotopía.

\subsection{Reflexividad}
Para $f\in\mathcal{C}(X,Y)$ definimos
\[
H(x,t)=f(x).
\]
Entonces $H(x,0)=f(x)$ y $H(x,1)=f(x)$, y $H$ es continua. Por tanto, $f\simeq f$.

\subsection{Simetría}
Si $H$ es una homotopía de $f$ a $g$, definimos
\[
K(x,t)=H(x,1-t).
\]
Entonces
\[
K(x,0)=H(x,1)=g(x),\qquad K(x,1)=H(x,0)=f(x),
\]
y $K$ es continua. Luego, $g\simeq f$.

\subsection{Transitividad}
Si $H$ es una homotopía de $f$ a $g$ y $G$ es una homotopía de $g$ a $h$, definimos $F:X\times[0,1]\to Y$ por pegado:
\[
F(x,t)=
\begin{cases}
H(x,2t), & 0\le t\le \tfrac12,\\[4pt]
G(x,2t-1), & \tfrac12\le t\le 1.
\end{cases}
\]
Se cumple:
\[
F(x,0)=H(x,0)=f(x),\qquad F(x,1)=G(x,1)=h(x),
\]
y en $t=\tfrac12$ coinciden porque
\[
H(x,1)=g(x)=G(x,0).
\]
Por el lema de pegado, $F$ es continua, así que $f\simeq h$.

\medskip
\noindent Conclusión: $\simeq$ es una relación de equivalencia sobre $\mathcal{C}(X,Y)$.

\section{Homotopía de caminos con extremos fijos (mención útil)}
Sean $\gamma_0,\gamma_1:[0,1]\to X$ caminos con los mismos extremos:
\[
\gamma_0(0)=\gamma_1(0),\qquad \gamma_0(1)=\gamma_1(1).
\]
Decimos que son \textbf{homotópicos con extremos fijos} si existe una aplicación continua
\[
H:[0,1]\times[0,1]\to X
\]
tal que
\[
H(s,0)=\gamma_0(s),\qquad H(s,1)=\gamma_1(s),
\]
y además mantiene fijos los extremos:
\[
H(0,t)=\gamma_0(0),\qquad H(1,t)=\gamma_0(1)\quad \text{para todo }t\in[0,1].
\]
Esta relación también es una relación de equivalencia y es base para definir el grupo fundamental $\pi_1(X)$.

\section{Conexión con TDA: ejemplo simple con Vietoris--Rips}
La idea en TDA es aproximar la ``forma'' de datos (puntos) construyendo complejos simpliciales a diferentes escalas $\varepsilon$
y calculando invariantes (por ejemplo, homología y números de Betti), que son invariantes por homotopía.

\subsection{Círculo unitario y puntos equiespaciados}
Tomamos $6$ puntos en el círculo unitario $S^1$:
\[
\theta_j=\frac{2\pi j}{6},\qquad
p_j=(\cos\theta_j,\sin\theta_j),\qquad j=0,1,2,3,4,5.
\]
\textbf{Interpretación de ``pasos'':} desde $p_j$, moverse $k$ pasos significa ir a $p_{j+k}$ (índices módulo $6$).

\subsection{Distancias por pasos}
El ángulo entre $p_j$ y $p_{j+k}$ es
\[
\Delta\theta=\theta_{j+k}-\theta_j=\frac{2\pi k}{6}.
\]
En un círculo de radio $1$, la cuerda correspondiente tiene longitud
\[
d(k)=2\sin\!\left(\frac{\Delta\theta}{2}\right)
=2\sin\!\left(\frac{k\pi}{6}\right).
\]
Por tanto:
\[
d(1)=1,\qquad d(2)=\sqrt3,\qquad d(3)=2.
\]

\subsection{Complejo de Vietoris--Rips $VR_\varepsilon$}
Dado un conjunto finito de puntos, el complejo $VR_\varepsilon$ se define así:
\begin{itemize}[leftmargin=*]
  \item Se agrega una arista $(p_i,p_j)$ si $\norm{p_i-p_j}\le \varepsilon$.
  \item Se agrega un triángulo $(p_i,p_j,p_k)$ si las tres distancias por pares son $\le\varepsilon$.
  \item En general, se agrega un símplice cuando todas las distancias por pares cumplen $\le\varepsilon$.
\end{itemize}

\subsection{Dos escalas y números de Betti}
\paragraph{Caso A: $1<\varepsilon<\sqrt3$ (ej. $\varepsilon=1.2$).}
Como $d(1)=1\le\varepsilon$, se conectan vecinos. Como $d(2)=\sqrt3>\varepsilon$, no se conectan puntos a dos pasos.
El complejo es un ciclo (hexágono), con un agujero:
\[
\betti_0=1,\qquad \betti_1=1.
\]
Intuición: $VR_\varepsilon$ ``se parece'' a un círculo ($VR_\varepsilon\simeq S^1$).

\paragraph{Caso B: $\sqrt3<\varepsilon<2$ (ej. $\varepsilon=1.75$).}
Ahora también se conectan puntos a dos pasos porque $d(2)=\sqrt3\le\varepsilon$.
Entonces para cada triple consecutivo $(p_j,p_{j+1},p_{j+2})$ aparecen las tres aristas y, por tanto, un triángulo que rellena el ciclo.
El agujero desaparece:
\[
\betti_0=1,\qquad \betti_1=0.
\]

\subsection{Comentario clave}
En TDA se estudia cómo cambian estos rasgos al variar $\varepsilon$ (persistencia). Como la homología es invariante por homotopía,
rasgos que persisten suelen interpretarse como estructura real y no ruido.

\end{document}
