\documentclass[12pt]{article}

% =========================
% Paquetes
% =========================
\usepackage[spanish]{babel}
\usepackage[utf8]{inputenc}
\usepackage[T1]{fontenc}
\usepackage{amsmath,amssymb,amsthm}
\usepackage{geometry}
\usepackage{hyperref}
\usepackage{enumitem}
\usepackage[spanish]{babel}
\shorthandoff{"}

\usepackage{tikz-cd}
\geometry{margin=2.5cm}

% =========================
% Entornos
% =========================
\newtheorem{definition}{Definición}
\newtheorem{remark}{Observación}

% =========================
% Documento
% =========================
\title{Categorías y Funtores: Mapa Jerárquico y Diagramas con \texttt{tikz-cd}\\
\large (con ejemplos para LLMs, Multi-Head y TDA)}
\author{}
\date{}

\begin{document}

\maketitle
\tableofcontents
\newpage

% ======================================================
\section{Introducción}
Este documento reúne un mapa jerárquico de tipos de categorías y, sobre todo,
un conjunto de diagramas conmutativos listos en \texttt{tikz-cd} para modelar
pipelines tipo LLM, multi-head attention y TDA.

% ======================================================
\section{Fundamentos: categoría, conmutatividad y funtores}

\subsection{Definición de categoría}
\begin{definition}
Una \textbf{categoría} $\mathcal{C}$ consiste en:
\begin{itemize}
    \item una clase de objetos $\mathrm{Ob}(\mathcal{C})$;
    \item para cada par $A,B$, un conjunto de morfismos $\mathrm{Hom}_{\mathcal{C}}(A,B)$;
    \item composición $\circ : \mathrm{Hom}(B,C)\times \mathrm{Hom}(A,B)\to \mathrm{Hom}(A,C)$;
    \item identidades $\mathrm{id}_A\in \mathrm{Hom}(A,A)$,
\end{itemize}
sujeto a asociatividad e identidad.
\end{definition}

\subsection{Diagrama conmutativo (el cuadrado básico)}
Un cuadrado
{\shorthandoff{"}
\[
\begin{tikzcd}
A \arrow[r,"f"] \arrow[d,"g"'] & B \arrow[d,"h"] \\
C \arrow[r,"k"']              & D
\end{tikzcd}
\]
}

\emph{conmuta} si
\[
h\circ f = k\circ g.
\]

\subsection{Definición de funtor (covariante)}
\begin{definition}
Un \textbf{funtor} $F:\mathcal{C}\to \mathcal{D}$ asigna:
\begin{itemize}
\item a cada objeto $A\in \mathcal{C}$ un objeto $F(A)\in \mathcal{D}$;
\item a cada morfismo $f:A\to B$ un morfismo $F(f):F(A)\to F(B)$;
\end{itemize}
de modo que $F(\mathrm{id}_A)=\mathrm{id}_{F(A)}$ y $F(g\circ f)=F(g)\circ F(f)$.
\end{definition}

\subsection{Funtor contravariante}
Un funtor contravariante es $F:\mathcal{C}^{op}\to \mathcal{D}$, es decir invierte flechas.

% ======================================================
\section{Mapa jerárquico de tipos de categorías}
\subsection{Mapa visual (alto nivel)}
{\shorthandoff{"}
\[
\begin{tikzcd}[row sep=large, column sep=huge]
\text{Categoría básica}
  \arrow[d, dashed, "\text{añadir estructura}"] \\
\text{Familias principales}
\end{tikzcd}
\]

\subsection{Árbol (familias principales)}
\[
\begin{tikzcd}[row sep=large, column sep=large]
& \textbf{Categorías (base)} \arrow[dl] \arrow[d] \arrow[dr] & \\
\textbf{Orden} \arrow[d] & \textbf{Algebra} \arrow[d] & \textbf{Geometría} \arrow[d] \\
\text{Preorden/Poset} & \text{Aditiva/Abeliana} & \text{Top/Métrica/Man} \\
& \textbf{Tensor} \arrow[d] & \textbf{Homológica} \arrow[d] \\
& \text{Monoidal/Cerrada} & \text{Triangulada/Derivada} \\
& \textbf{Lógica} \arrow[d] & \textbf{Orden superior} \arrow[d] \\
& \text{CCC/Topos} & \text{2-cat, $n$-cat, $\infty$-cat}
\end{tikzcd}
\]

\subsection{Notas mínimas por familia}
\begin{itemize}
\item \textbf{Orden}: categorías delgadas (preorden/poset).
\item \textbf{Álgebra}: aditivas/abelianas (núcleos, cokernels, exactitud).
\item \textbf{Tensor}: monoidales (producto $\otimes$), monoidales cerradas (hom interno).
\item \textbf{Geometría}: Top, Metric, Man.
\item \textbf{Homológica}: simpliciales, trianguladas, derivadas.
\item \textbf{Lógica}: CCC y topos.
\item \textbf{Orden superior}: 2-categorías, $n$-categorías, $\infty$-categorías.
\end{itemize}

% ======================================================
\section{Diagramas listos para ML: LLM, TDA, multi-head, adjunciones}

\subsection{LLM como composición (Set $\to$ Vect $\to$ Prob)}
Interpretación estándar:
\[
LLM = Dec\circ T \circ Emb.
\]
Diagrama:
\[
\begin{tikzcd}[column sep=huge]
\mathbf{Set}
  \arrow[r,"Emb"]
&
\mathbf{Vect}
  \arrow[r,"T"]
&
\mathbf{Vect}
  \arrow[r,"Dec"]
&
\mathbf{Prob}
\end{tikzcd}
\]

Con la composición explícita:
\[
\begin{tikzcd}[column sep=huge]
\mathbf{Set}
  \arrow[rrr, bend left=18, "LLM"]
  \arrow[r,"Emb"]
&
\mathbf{Vect}
  \arrow[r,"T"]
&
\mathbf{Vect}
  \arrow[r,"Dec"]
&
\mathbf{Prob}
\end{tikzcd}
\]

\subsection{Naturaleza de ``conmutar'' en el pipeline}
Si tienes una transformación $f:x\to y$ en $\mathbf{Set}$, el requisito funtorial dice:
\[
LLM(f) = Dec(T(Emb(f))).
\]
Diagrama:
\[
\begin{tikzcd}[row sep=large, column sep=huge]
x \arrow[r,"f"] \arrow[d,"LLM"'] & y \arrow[d,"LLM"] \\
LLM(x) \arrow[r,"LLM(f)"'] & LLM(y)
\end{tikzcd}
\]

\subsection{LLM + TDA (vector topológico)}
Pipeline típico:
\[
\mathbf{Vect}\to \mathbf{Metric}\to \mathbf{SimpComp}\to \mathbf{Ab}\to \mathbf{Vect}.
\]
Diagrama:
\[
\begin{tikzcd}[column sep=large]
\mathbf{Vect}
  \arrow[r,"d\ \text{(métrica)}"]
&
\mathbf{Metric}
  \arrow[r,"VR_\varepsilon\ \text{(Rips)}"]
&
\mathbf{SimpComp}
  \arrow[r,"H_k\ \text{(homología)}"]
&
\mathbf{Ab}
  \arrow[r,"\Phi\ \text{(vectorizar)}"]
&
\mathbf{Vect}
\end{tikzcd}
\]

\subsection{Funtor de olvido (Metric $\to$ Top)}
La métrica induce topología (olvidas distancias exactas, conservas abiertos):
\[
\begin{tikzcd}[column sep=huge]
\mathbf{Metric} \arrow[r,"Forget"] & \mathbf{Top}
\end{tikzcd}
\]

\subsection{Adjunción (Free $\dashv$ Forget) como diagrama}
Una forma compacta de recordarlo:
\[
\mathrm{Hom}_{\mathbf{Vect}}(Free(S),V)\ \cong\ \mathrm{Hom}_{\mathbf{Set}}(S,Forget(V)).
\]
Diagrama:
\[
\begin{tikzcd}[row sep=large, column sep=huge]
\mathbf{Set}
  \arrow[r, shift left=1.2, "Free"]
&
\mathbf{Vect}
  \arrow[l, shift left=1.2, "Forget"]
\end{tikzcd}
\]
Y se anota $Free \dashv Forget$.

\subsection{Multi-Head: suma directa de cabezas y proyección}
Modelado estructural:
\[
V \xrightarrow{\Delta} V^{\oplus h} \xrightarrow{W_O} V,
\quad
\Delta(x)=(h_1(x),\dots,h_h(x)).
\]
Diagrama:
\[
\begin{tikzcd}[column sep=huge]
V
  \arrow[r,"\Delta = h_1 \oplus \cdots \oplus h_h"]
&
V^{\oplus h}
  \arrow[r,"W_O"]
&
V
\end{tikzcd}
\]

También puedes dibujar ``fork'' (una cabeza arriba y otra abajo, sugerencia visual):
\[
\begin{tikzcd}[row sep=large, column sep=large]
& V \arrow[dl,"h_1"'] \arrow[dr,"h_h"] & \\
V \arrow[dr, dotted] & & V \arrow[dl, dotted] \\
& V^{\oplus h} \arrow[d,"W_O"] & \\
& V &
\end{tikzcd}
\]
(Las punteadas solo indican ``agrupación'' en la suma directa.)

\subsection{Estructura monoidal: coherencia con tensor}
Si $T$ fuera monoidal (idealización), esperas compatibilidad:
\[
T(V\otimes W)\ \cong\ T(V)\otimes T(W).
\]
Diagrama (como cuadrado de coherencia):
\[
\begin{tikzcd}[row sep=large, column sep=huge]
V\otimes W \arrow[r,"T"] \arrow[d, "\cong"'] & T(V\otimes W) \arrow[d, dashed, "\text{estructura monoidal}"] \\
V\otimes W \arrow[r, dashed, "T(V)\otimes T(W)"'] & T(V)\otimes T(W)
\end{tikzcd}
\]

\subsection{Optimizador como endomorfismo (dinámica de entrenamiento)}
Un optimizador define una iteración:
\[
\theta_{t+1}=\Phi(\theta_t),\qquad \Phi:\Theta\to \Theta.
\]
Diagrama:
\[
\begin{tikzcd}[column sep=huge]
\Theta \arrow[r,"\Phi"] & \Theta \arrow[r,"\Phi"] & \Theta \arrow[r,"\Phi"] & \cdots
\end{tikzcd}
\]
Si $\Theta$ es variedad (Man) o espacio vectorial (Vect) depende de tu formalización.

\subsection{Diagrama completo ``stack ML moderno'' (resumen)}
Uniendo idea de embeddings, geometría y TDA:
\[
\begin{tikzcd}[column sep=large]
\mathbf{Set}
  \arrow[r,"Emb"]
&
\mathbf{Vect}
  \arrow[r,"T\ \text{(Transformer)}"]
&
\mathbf{Vect}
  \arrow[r,"d"]
&
\mathbf{Metric}
  \arrow[r,"VR_\varepsilon"]
&
\mathbf{SimpComp}
  \arrow[r,"H_k"]
&
\mathbf{Ab}
  \arrow[r,"\Phi"]
&
\mathbf{Vect}
  \arrow[r,"Dec"]
&
\mathbf{Prob}
\end{tikzcd}
\]

% ======================================================
\section{Tips prácticos de \texttt{tikz-cd}}
\begin{itemize}
\item Direcciones: \texttt{r,l,u,d,dr,dl,ur,ul}.
\item Etiqueta al otro lado: \texttt{"f"'}.
\item Curvar flecha: \texttt{bend left=20} o \texttt{bend right=20}.
\item Punteada: \texttt{dashed}.
\item Ajustar espacios: \texttt{[row sep=large, column sep=huge]}.
\item Mono/Epi: \texttt{hookrightarrow}, \texttt{twoheadrightarrow}.
\end{itemize}

% ======================================================
\section{Cierre}
LLM como funtor compuesto, TDA como cadena de funtores,
multi-head como suma directa + proyección,
adjunciones (Free $\dashv$ Forget), coherencia monoidal y optimización como dinámica.

\end{document}
